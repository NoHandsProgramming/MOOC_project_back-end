\documentclass[a4paper, 12pt]{article}

\usepackage{xspace}
\usepackage{listings}
\usepackage{hyperref}

\newcommand{\projectTitle}{\texttt{MOOC Project} }

\begin{document}

\title{Architecture Document for Backend Division of \projectTitle}
\author{Andrew Lalis\\ andrewlalisofficial@gmail.com}
\date{Last modified on\\ \today}
\maketitle

\setcounter{tocdepth}{3}
\tableofcontents

\section{Introduction}
	\subsection{Problem}
		The problem that many programmers face is finding a project to collaborate on, in a way that doesn't leave them feeling intimidated by a lack of skill or inability to fully grasp the scope of a large, multi-person project.

	\subsection{Solution}
		To solve the aforementioned problem, we propose to introduce a web application to allow users to easily search for projects to collaborate on. The goal is to allow the user a friendly and quick way to find something to work on, with the skillset they already have. The point is to make it easier for new programmers to gain experience working on larger projects, in a way that makes it easier both for project managers and the individual developers.

\section{User Stories}
	Here, some common experiences users should expect are written, from their point of view. The purpose is to prepare the application in such a way that a user can interact with it in a natural way, without struggling with overly complex functionality and unintuitive design choices.

	Due to the nature of this part of the project, specifically the back-end architecture, many of the stories will be rather abstract, and it is the responsibility of the developers to design the application so that it fulfills these requirements in an efficient and timely manner.

	\subsection{Critical User Stories}
		The following user stories represent the basic requirements the project must fulfill to be considered complete. Without fulfilling \textit{each} of these requirements, the project will not work.
		\begin{itemize}
			\item As a user, I want to see all the projects I'm currently a part of.
			\item As a user, I want to be able to let the site know that I am skilled with certain technologies.
			\item As a user, I want to see projects that are relevant to the skills I have shared with the site.
			\item As a project manager, I want to be able to easily create a new project, and set a list of skills as a set of requirements to filter out the many unqualified applicants.
			\item As a project manager, I want a way to chat with the users in my project easily.
			\item As a project manager, I would like to chat with a potential member before they join the project, as an added layer of security/scrutiny.
		\end{itemize}
		
	\subsection{Important User Stories}
		These stories represent preferences which, while they are not 100\% necessary for an operational project, significantly enhance the user experience.
		\begin{itemize}
			\item ENTER USER STORY HERE
		\end{itemize}
	
	\subsection{Relevant User Stories}
		Stories in this section describe details and design choices which, while they do not impact the viability of the project, may have influence on its success when launched to a public audience.
		\begin{itemize}
			\item ENTER USER STORY HERE
		\end{itemize}

\section{Relational Database Design}
	To store all the relevant data for such a broad project, we propose to use a relational \textsc{SQL} database to store information such as the users, project metadata, and skills for all the users. We chose a relational database, as opposed to, for example, a JSON object database like those used by many Google applications, because of the number of connections between different entities in the database, and the ease with which \textsc{SQL} can manipulate and fetch this related information.

\section{Glossary}
	Throughout the architecture document, several key words or phrases are used, many specific to the scope of this project. As such, those terms are defined below.

	\begin{enumerate}
		\item \emph{The Project} - The \projectTitle as a whole. It refers to the collective efforts of both the front- and back-end projects.

		\item \emph{User} - Someone who uses the project for its intended purpose; to find other users to collaborate with, or to host a project so that other users may join it.

		\item \emph{Project Manager} - A subset of users who have created their own projects.

		\item \emph{Project} - A digital representation of a software collaboration project, lead by a project manager, and a list of users who are members of the project.

	\end{enumerate}

\end{document}